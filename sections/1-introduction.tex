This project aims to implement a Simultaneous Localization and Mapping (SLAM) algorithm on a TurtleBot3 Waffle Pi robot using the Robot Operating System 2 (ROS2) framework for the purpose of autonomous environment mapping, localization, and navigation. In other words, the goal will be for the robot to move in an environment from one point to another while avoiding obstacles. SLAM is important for autonomous vehicles as it allows a vehicle to map a room and localize itself in it at the same time.

This paper utilizes a variety of robotics terminology. The term ``pose'' refers to an object's position and orientation in space, typically represented by a three-dimensional vector ($x,y,\theta$). Odometry encompasses the measurements of an object's travel distance and direction, often achieved through the use of wheel encoders \parencite{corkeRoboticsVisionControl2023}. Dead reckoning is the process of estimating an object's pose based on the information obtained from odometry data. Put differently, this navigation method doesn't rely on any observations of the environment.

Many optimization algorithms are used in this project. These algorithms often need to use the gradient to iteratively optimize functions. For a multivariable function, the gradient at a given point is a vector with a direction pointing in the direction of fastest increase and a magnitude representing the slope in that direction \parencite{Gradient2024}. For a point $p = (x_1,\hdots, x_n)$ in $\mathbb{R}^n$, the gradient is given by
\[
    \nabla f(p)=\begin{bmatrix}
        \frac{\partial f}{\partial x_1}(p) \\[6pt]
        \vdots                             \\[6pt]
        \frac{\partial f}{\partial x_n}(p)
    \end{bmatrix}.
\]

As of the time of writing this paper, the robot is able to map and localize itself in its environment. There is still an issue with the global path planner due to over-inflation of the obstacles, but it can be fixed by changing the inflation parameters. Otherwise, when dead reckoning to a target position, a significant discrepancy between the estimated and actual robot location is observed. This is caused by wheel slippage and velocities being rounded to 0.